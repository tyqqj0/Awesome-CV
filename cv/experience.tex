%-------------------------------------------------------------------------------
%	SECTION TITLE
%-------------------------------------------------------------------------------
\cvsection{Research Experience}


%-------------------------------------------------------------------------------
%	CONTENT
%-------------------------------------------------------------------------------
\begin{cventries}

  %---------------------------------------------------------
  \cventry
  {Neural Computing and Intelligent Perception (NCIP)}
  {Leveraging Medical Knowledge to Guide Transfer Learning for Cerebrovascular Segmentation}
  {Beijing, China}
  {Sep. 2024 - Present}
  {
    \begin{cvitems}
      \item {Constructing medical image large model agents by combining text-vision tokenizers with fundamental segmentation models, attempting to integrate medical expert prior knowledge into the model training process to guide few-shot transfer learning for cerebrovascular segmentation tasks.}
      \item {\textbf{Expected outcome:} Paper to be published at MICCAI 2025.}
    \end{cvitems}
  }

  %---------------------------------------------------------
  \cventry
  {Neural Computing and Intelligent Perception (NCIP)}
  {Developing Noise-Robust Deep Learning Methods for Hard Samples}
  {Beijing, China}
  {Dec. 2023 - Sep. 2024}
  {
    \begin{cvitems}
      \item {Discovered that current noisy label robust learning methods fail on hard samples. Proposed an iterative pseudo-label estimation method using optimal sample weighting to improve model performance on real-world noisy label datasets with a large number of hard samples.}
      \item {\textbf{Outcome:} Paper submitted to NeurIPS 2024.}
    \end{cvitems}
  }

  %---------------------------------------------------------
  \cventry
  {Neural Computing and Intelligent Perception (NCIP)}
  {Enhancing Small Sample Cerebrovascular Segmentation Accuracy via MRI-CT Transfer Learning}
  {Beijing, China}
  {Sep. 2022 - Dec. 2022}
  {
    \begin{cvitems}
      \item {Introduced transfer learning methods into CT cerebrovascular segmentation tasks. Utilized 3D-UNet variants pre-trained on MRI data to transfer deep knowledge of cerebrovascular segmentation learned on MRI to CT data, achieving more accurate CT cerebrovascular segmentation models.}
      \item {\textbf{Outcome:} Papers published at IEEE ISBI 2023 and in Computers in Biology and Medicine journal.}
    \end{cvitems}
  }

  %---------------------------------------------------------
  \cventry
  {Neural Computing and Intelligent Perception (NCIP)}
  {Collaborating with Hospitals to Build High-Quality Cerebrovascular CTA Datasets}
  {Beijing, China}
  {May. 2022 - Dec. 2022}
  {
    \begin{cvitems}
      \item {Collaborated with Beijing 301, 304, and China-Japan Friendship Hospital to preprocess and annotate cerebrovascular CTA data, constructing high-quality datasets for cerebrovascular segmentation and cerebral hematoma detection.}
      \item {\textbf{Outcome:} High-quality multi-task, multi-center cerebrovascular datasets.}
    \end{cvitems}
  }
  %---------------------------------------------------------
\end{cventries}
